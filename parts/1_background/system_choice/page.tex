\chapter{System Choice}
Define the system in its context problem domain + application domain.\\
Using the F.A.C.T.O.R system.\\
F: Functionality - System functions that support the AP tasks.
\\\\
A: Application domain - The organization that administrates the problem domain.
Where the user is
\\\\
C: Condtion - The conditions under which the system will be developed
\\\\
T: Technology - Both the technology used to develop the system and the technology which the system will run.
\\\\
O: Objects
\\\\
R: Responsibility

\section{System definition}
\subsection*{Situation}
Describe the situation based on the context of the system. Described in a rich picture.\\
Which is the described by its focus, entities, procceses and structure.

\begin{figure}[h]
    \centering
    \includegraphics*[width=\linewidth]{parts/1_background/system_choice/figure/rich_picture.png}
\end{figure}
\subsection*{Ideas}
Examples - eg. Study preexisting systems.

\subsection*{System definition}
Not to be confused with \textit{system}. A concise description of a computerrized system expressed in natural language.

\subsection*{Context}

\section{Problem Domain Analysis}
The result of a problem domain analysis is a class diagram describing classes and structure.

\subsection*{Event}
An event in the problem analysis is:\\
 - Atomic\\
 - An incident involving one or more objects\\
 - Instantaneous\\

 \subsection*{System Model}
 The system's model of the problem domain\\
 - Represents the state of the problem domain\\
 - Provides information to users in the application domain about the problem domain\\
 