\chapter{Method}
\section*{Concepets}
\begin{center}
    \begin{tabular}{ l l }
        \hline
        Object & Entity with identity, state and behaviour\\\hline
        Class  & Describes a collection of objects sharing structure, \\ & behavioural patterns and attributes\\\hline
        Problem domain & Part that is administrated, monitored or controlled by a system \\\hline
        Application domain & The organization that administrates the problem domain\\ & Where the user is\\ \hline
        System & A collection components that implements \\ &  modeling requirements, functions and interfaces \\\hline        
        Context & Problem domain and application domain\\\hline
    \end{tabular}
\end{center}
\todo{Describe problem- and application domain better}
\todo{Der er eksempler bag i bogen}

\section*{Problem domain}
Class structure and behaviour

\section*{Application domain}
Usage functions and interfaces

\section*{Method}
Purpose, concepts, principles and results.

\section{Objects and classes}
\subsection*{Objects - \it{Entity with identity, state and behaviour}}
Each object serves as a seperate function. The object could be a customer, where specific people are treated as customers. The object contains that specific customers identity, state and behaviour

\subsection*{Class - \it{Describes a collection of objects sharing structure}}
The class contains multiple objects, meaning a customer class will contain multiple data points. The class also contains multiple different customers and their data points.\\
When describing a class it's important to choose the right granularity. Gravel pit should not be described by the individual grains of sand, instead by the type, whereas a warehouse the individual packages should be described.\\

\subsection*{Analysis - outside the system}
In analysis the object's behaviour is described by its events it performece and experiences that happens in definite points in time. Eg. customers ordering and shipping goods.\\

\subsection*{Design - inside the system}
In design the object's behaviour is described by the operations it can perform and make available to other objects in the system. Eg. add order etc. \\
This allows the update of eg. the customers object state. The design object encapsulates the internal representation of the object state through its operations.\\

\section{Principles}
The 4 principles:
\\\\
\textit{Model the context} -
Useful systems fit the context, so model both application and problem domain during analysis and design.
\\\\
\textit{Emphasize the architecture} -
Understandable architecture makes collaboration between programmers and designers possible. Flexible architecture makes modifications and improvements affordable
\\\\
\textit{Reuse patterns} -
Building on well-established ideas and pretested components
\\\\
\textit{Tailor the method to suit specific project} -
Must be tailored to the specific needs of the analysis and design situation 

\section*{Model}
Model is a representation of the state in the problem domain. How often the model is updated is a design decision\\
Problem domain -> model -> application domain.
